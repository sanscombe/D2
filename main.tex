\documentclass[11pt, a4paper]{amsart}

\usepackage{amsmath,amssymb,amsfonts,amsthm}
\usepackage[margin=2cm]{geometry}
\usepackage[]{setspace}\onehalfspacing
\usepackage[usenames,dvipsnames]{color}
\usepackage{hyperref}
\hypersetup{
    colorlinks=true,
    linkcolor= blue,
    citecolor=blue
}
\usepackage{fixmath}
\usepackage[inline]{showlabels}
\usepackage{comment,stmaryrd}
\usepackage{aliascnt}
% 
\usepackage{enumitem}
\newenvironment{enumeratenice}
{
    \begin{enumerate}%
    [label={\bf\arabic*}., itemsep=1ex,leftmargin=1.5cm]%2
}
{
  \end{enumerate}
}
%
\usepackage[T1]{fontenc}
\usepackage[utf8]{inputenc}
% \usepackage{libertine}
% 
\usepackage{tikz}
\usetikzlibrary{matrix,arrows,decorations.pathmorphing}
\usepackage[normalem]{ulem}

\usepackage[framemethod=TikZ]{mdframed}
\usepackage{tikzpagenodes}

\mdfsetup{skipabove=10pt,skipbelow=10pt}
\mdfdefinestyle{axioms}{%
    % nobreak=true,
    linecolor=pink,
    outerlinewidth=1pt,
    roundcorner=10pt,
    innertopmargin=0pt,
    innerbottommargin=10pt,
    innerrightmargin=20pt,
    innerleftmargin=20pt,
    backgroundcolor=gray!10!white
}

\theoremstyle{plain}
\newtheorem{theorem}{Theorem}%[section]

%%Simulated counters
\newaliascnt{proposition}{theorem}
\newaliascnt{lemma}{theorem}
\newaliascnt{corollary}{theorem}
%\newaliascnt{claim}{theorem}
\newaliascnt{fact}{theorem}
\newaliascnt{observation}{theorem}
\newaliascnt{conjecture}{theorem}
\newaliascnt{definition}{theorem}
\newaliascnt{example}{theorem}
\newaliascnt{question}{theorem}
\newaliascnt{remark}{theorem}
\newaliascnt{property}{theorem}
\newaliascnt{construction}{theorem}
\newaliascnt{setting}{theorem}
\newaliascnt{axioms}{theorem}

%% Secondary theorem environments
\theoremstyle{plain}
% \newmdtheoremenv[style=axioms]{axioms}[theorem]{Axioms}
\newtheorem{axioms}[axioms]{Axioms}
\newtheorem{proposition}[proposition]{Proposition}
\newtheorem{lemma}[lemma]{Lemma}
\newtheorem{corollary}[corollary]{Corollary}
\newtheorem{claim}{Claim}[theorem]
\newtheorem{fact}[fact]{Fact}
\newtheorem{observation}[observation]{Observation}
\newtheorem{conjecture}[conjecture]{Conjecture}
\theoremstyle{definition}
\newtheorem{definition}[definition]{Definition}
\newtheorem{example}[example]{Example}
\newtheorem{question}[question]{Question}
\theoremstyle{remark}
\newtheorem{remark}[remark]{Remark}
\newtheorem{property}[property]{Property}
\newtheorem{construction}[construction]{Construction}
\newtheorem{setting}[setting]{Setting}

%%Reset the simulated counters
\aliascntresetthe{proposition}
\aliascntresetthe{lemma}
\aliascntresetthe{corollary}
%\aliascntresetthe{claim}
\aliascntresetthe{fact}
\aliascntresetthe{observation}
\aliascntresetthe{conjecture}
\aliascntresetthe{definition}
\aliascntresetthe{example}
\aliascntresetthe{question}
\aliascntresetthe{remark}
\aliascntresetthe{property}
\aliascntresetthe{construction}
\aliascntresetthe{setting}
\aliascntresetthe{axioms}

%%Custom labels%%
\providecommand{\propositionautorefname}{Proposition}
\providecommand{\lemmaautorefname}{Lemma}
\providecommand{\corollaryautorefname}{Corollary}
\providecommand{\claimautorefname}{Claim}
\providecommand{\factautorefname}{Fact}
\providecommand{\observationautorefname}{Observation}
\providecommand{\conjectureautorefname}{Conjecture}
\providecommand{\definitionautorefname}{Definition}
\providecommand{\exampleautorefname}{Example}
\providecommand{\questionautorefname}{Question}
\providecommand{\remarkautorefname}{Remark}
\providecommand{\propertyautorefname}{Property}
\providecommand{\constructionautorefname}{Construction}
\providecommand{\settingautorefname}{Setting}
\providecommand{\axiomsautorefname}{Axioms}


\newcommand\sqbr{[\;,\,]}
\newcommand\betw{\mathbf{betw}}
\newcommand\coll{\mathbf{coll}}
\newcommand\Coll{\mathbf{Coll}}
\newcommand\copl{\mathbf{copl}}
\newcommand\Copl{\mathbf{Copl}}
\newcommand\para{\mathbf{parallel}}


%%% HEADER %%%
\title[D2]{D2}
\author{Sylvy Anscombe}
\address{Jeremiah Horrocks Institute, University of Central Lancashire, Preston PR1 2HE, United Kingdom}
\email{sanscombe@uclan.ac.uk}
\thanks{\today}

\begin{document}

\maketitle


\begin{enumeratenice}
\item[{\bf(A1)}]
\item[{\bf(A2)}]
\item[{\bf(A3)}]
\item[{\bf(A4)}]
\item[{\bf(A5)}]
\end{enumeratenice}





\vfill




%%% ACKNOWLEDGEMENTS %%%
% \section*{Acknowledgements}

% The author would like to extend our thanks to
% XXXXXX
% for numerous helpful conversations.
% {\color{red}anyone else?}

% Some of this work was completed while we were participating in the
% {\em 
% Model Theory, Combinatorics and Valued fields}
% trimester at the Institut Henri Poincar\'{e},
% and we would like to extend our thanks to the organisers.
% Sylvy Anscombe was also partially supported by The Leverhulme Trust
% under grant RPG-2017-179.


% %%% BIBLIOGRAPHY %%%
% % \def\bibfont{\footnotesize}
% \bibliographystyle{plain}
% % \begin{singlespacing}
% \begin{thebibliography}{AAA00a}

% % \bibitem[AJ18]{AJ18}
% % Sylvy Anscombe and Franziska Jahnke.
% % \newblock {\em Dreaming of NIP fields.}
% % \newblock Manuscript, 2018.

% % \bibitem[B\'{e}l99]{Bel99}
% % Luc B\'{e}lair.
% % \newblock {\em Types dans les corps valu\'{e}s munis d'applications coefficients.}
% % \newblock Illinois J.~Math., 43(2):410--425, 1999.

% % \bibitem[CC56]{CC56}
% % Henri Cartan and Claude Chevalley.
% % \newblock {\em G\'{e}om\'{e}trie Alg\'{e}brique:}
% % \newblock S\'{e}minaire H.~Cartan et C.~Chevalley, ENS, 8e ann\'{e}e, 1955--1956.

% % \bibitem[Che43]{Che43}
% % Claude Chevalley.
% % \newblock {\em On the theory of local rings.}
% % \newblock {Ann.~Math.}, 44(4):690--708, 1943.

% % \bibitem[Coh46]{Coh46}
% % I.~S.~Cohen.
% % \newblock {\em On the structure and ideal theory of complete local rings.}
% % \newblock Trans.~Amer.~Math.~Soc., 59:54--106, 1946.

% % % \bibitem[Del82]{Del82}
% % % Fran\c{c}oise Delon.
% % % \newblock {\em Quelques propri\'{e}t\'{e}s des corps valu\'{e}s en th\'{e}orie des mod\`{e}les.}
% % % \newblock Th\`{e}se de Doctorat d'\'{E}tat, Universit\'{e} Paris VII, 1982.

% % \bibitem[DM15]{DM15}
% % Jamshid Derakhshan and Angus Macintyre.
% % \newblock {\em Model completeness for henselian fields with finite ramification valued in a $\mathbb{Z}$-group}.
% % \newblock Manuscript (??), 2015.

% % \bibitem[HS34]{HS34}
% % H.~Hasse and F.~K.~Schmidt.
% % \newblock {\em Die Struktur diskret bewerteter K\"{o}rper.}
% % \newblock J.~reine angew.~Math., 170:4--63, 1934.

% % \bibitem[Kru37]{Kru37}
% % Wolfgang Krull.
% % \newblock {\em Beitr\"{a}ge zur Arithmetik kommutativer Integrit\"{a}tsberiche. III Zum Dimensionsbegriff der Idealtheorie.}
% % \newblock {Math.~Zeit.}, 42:745-766, 1937.

% % \bibitem[Kru38]{Kru38}
% % Wolfgang Krull.
% % \newblock {\em Dimensionstheorie in Stellenringen}.
% % \newblock J.~reine angew.~Math., 179:204--226, 1938.

% % \bibitem[Mac39a]{Mac39a}
% % Saunders Mac Lane.
% % \newblock {\em Modular fields. I. Separating transcendence bases.}
% % \newblock Duke Math.~J., 5:372--393, 1939.

% % \bibitem[Mac39b]{Mac39b}
% % Saunders Mac Lane.
% % \newblock {\em Steinitz field towers for modular fields.}
% % \newblock Trans.~Amer.~Math.~Soc., 46:23--45, 1939.

% % \bibitem[Mac39c]{Mac39c}
% % Saunders Mac Lane.
% % \newblock {\em Subfields and Automorphisms Groups of $p$-Adic Fields.}
% % \newblock {Ann.~Math.}, 40(2):423--442, 1939.

% % \bibitem[Mac40]{Mac40}
% % Saunders Mac Lane.
% % \newblock {\em Modular fields.}
% % \newblock Amer.~Math.~Monthly, 47:259--274, 1940.

% % \bibitem[PR84]{PR84}
% % Alexander Prestel and Peter Roquette.
% % \newblock {\em Formally $p$-adic fields.}
% % \newblock Lecture Notes in Logic, Springer, 1984.

% % \bibitem[Roq03]{Roq03}
% % Peter Roquette.
% % \newblock {\em History of Valuation Theory, Part I.}
% % \newblock Available online: {\href{https://www.mathi.uni-heidelberg.de/~roquette/hist_val.pdf}{\scriptsize\tt{https://www.mathi.uni-heidelberg.de/\textasciitilde roquette/hist\textunderscore val.pdf}}}, 2003.

% % \bibitem[Sch72]{Sc72}
% % Colette Schoeller
% % \newblock {\em Groupes affines, commutatifs, unipotents sur un corps non parfait.}
% % \newblock {Bull.~de la S.~M.~F.}, 100:241--300, 1972.

% % \bibitem[Ser79]{Ser79}
% % Jean-Pierre Serre.
% % \newblock {\em Local Fields} (English translation of {\em Corps locaux}).
% % \newblock Springer, 1979.

% % \bibitem[Ser00]{Ser00}
% % Jean-Pierre Serre.
% % \newblock {\em Local Algebra} (English translation of {\em Alg\`{e}bre Locale -- Multiplicit\'{e}s}).
% % \newblock Springer, 2000.

% % \bibitem[Tei36a]{Tei36a}
% % Oswald Teichm\"{u}ller.
% % \newblock {\em $p$-Algebren.}
% % \newblock {Deutsche Math.}, 1:362--388, 1936.

% % \bibitem[Tei36b]{Tei36b}
% % Oswald Teichm\"{u}ller.
% % \newblock {\em \"{U}ber die Struktur diskret bewerteter perfekter K\"{o}rper.}
% % \newblock In: {\em Nachrichten von der Gesellschaft der Wissenschaften zu G\"{o}ttingen, Mathematisch-Physikalishe Klasse: Fachgruppe 1, Nachrichten aus der Mathematik}, 1(10):152--161, 1936.

% % \bibitem[Tei37]{Tei37}
% % Oswald Teichm\"{u}ller.
% % \newblock {\em Diskret bewerteter perfekte K\"{o}rper mit unvollkommenem Restklassenk\"{o}rper.}
% % \newblock {J.~reine angew.~Math.}, 176:141--152, 1937.

% % \bibitem[vdD14]{vdD14}
% % Lou van den Dries.
% % \newblock {\em Lectures on the Model Theory of Valued Fields.}
% % \newblock In Lou van den Dries, Jochen Koenigsmann, H.\ Dugald Macpherson, Anand Pillay, Carlo Toffalori, Alex J.\ Wilkie,
% % {\em Model Theory in Algebra, Analysis and Arithmetic}, 55--157.
% % \newblock Lecture Notes in Mathematics, vol.\ 2111.
% % \newblock Springer, Berlin, Heidelberg, 2014.

% % \bibitem[Wei84]{Wei84}
% % Volker Weispfenning.
% % \newblock {\em Quantifier elimination and decision procedures for valued fields.}
% % \newblock In: M\"{u}ller G.H., Richter M.M.~(eds) {\em Models and Sets}. Lecture Notes in Mathematics, vol 1103. Springer, Berlin, Heidelberg, 1984.

% % \bibitem[Wit??]{Wit??}
% % Witt.
% % \newblock {\em ????????????}.
% % \newblock ????????, 19??.

% \end{thebibliography}
% % \end{singlespacing}
\end{document}